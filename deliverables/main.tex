\documentclass{article}

\usepackage{graphicx}
\usepackage[margin=1.0in]{geometry}
\usepackage[version=4]{mhchem}
\usepackage{siunitx}
\usepackage{amsmath}
\usepackage{listings}
\usepackage[square,numbers]{natbib}
\usepackage{hyperref}
\usepackage{booktabs}
\hypersetup{
    colorlinks=true,
    linkcolor=blue,
    filecolor=magenta,      
    urlcolor=cyan,
    pdftitle={Overleaf Example},
    pdfpagemode=FullScreen,
    }

\newcommand{\radlen}{\gram\per\cm^2}

\title{\vspace{-3em}{\bf Studying High-Energy Particle Cascades with One-Dimensional Monte-Carlo Simulations}}
\author{Matt Ketkaroonkul}
\date{\today}

\begin{document}
\maketitle

\begin{abstract}
    Write the abstract last.
    
\end{abstract}

\section{Introduction} % (fold)
\label{sec:Introduction}
Particle cascades are a topic of great interest in high energy physics. They occur as cosmic rays, collider interactions, and more.

This project will model a one-dimensional particle cascade of a \SI{1.0}{\giga\eV} photon as it interacts with a mass of lead (Pb). As the photon travels through the lead mass, the photon will undergo pair production, where the photon becomes a positron-electron pair. Each particle having half of the precursor photon's total energy. The charged particles are able to interact with the lead atoms themselves, undergoing ionization losses and bremsstrahlung (braking radiation.)

To describe particle cascades within matter, it is useful to define a quantity called radiation length, which has units equivalent to mass density multiplied by distance. In this paper, we will use the units \unit{\radlen}. Radiation length describes the mean length a high-energy particle travels before it loses energy by a factor of \( 1 / e \) for a given material that makes up a uniform medium. Thus, the density is the density of the medium.

Ionization losses scale linearly with radiation length of a given charged particle. On average, we will assume that a charged particle will lose \SI{2.0}{\mega\eV\per\radlen} of lead traveled through. Bremsstrahlung occurs for charged particles above a critical value of energy of \( E_c = \SI{6.9}{\mega\eV}\). A charged particle with an energy above \( E_c \) will lose 

% section Introduction (end)
\section{Implementation} % (fold)
\label{sec:Implementation}
\subsection{Exponential Distribution} % (fold)
\label{sub:Exponential Distribution}
An exponential distribution has the probability distribution function of
\begin{equation}
    P(x|X_{0}) = \frac{1}{X_{0}} \exp \left( -\frac{x}{X_{0}} \right) 
\end{equation}

where \( X_{0} \) is a scale factor, i.e. the radiation length of lead. In code, we will assume that we only have access to a random variable \( x \) sampled from a uniform distribution. We can sample an exponential distribution from a uniform distribution using \textit{Inverse Transform Sampling}, where
\begin{equation}
    x = F^{-1} (y) .
\end{equation}
\( F(x)= e^{-x / X_{0}} / X_{0} \) and \( y \) is the uniform distribution random variable \cite{bohm}. Of course, this principle applies for when \( x, y \) share a one-to-one mapping. We can then plot a histogram depicting this transformed distribution with the code in Section \ref{sub:Exponential Distribution Code},

\begin{figure}[htpb]
    \begin{center}
        \includegraphics[width=0.7\textwidth]{figures/exp_dist.pdf}
    \end{center}
    \caption{Exponential distribution histogram and PDF with \( X_{0} = 1 \).}\label{fig:exp-dist}
\end{figure}

% subsection Exponential Distribution (end)
\subsection{Monte-Carlo Procedure} % (fold)
\label{sub:Monte-Carlo Procedure}
To keep track of photons and charged particles, two \( n \times 3 \) arrays were initialized. Column 0 stores the energy of the particle, column 1 stores the \textit{initial} radiation length, column 2 stores the \textit{displacement} radiation length, and column 3 stores the ``state'' of the particle (see Table \ref{tab:particle-state} for more information).
\begin{table}
    \begin{center}
        \begin{tabular}[c]{l|l}
            \multicolumn{1}{c|}{\textbf{Enumeration}} & 
            \multicolumn{1}{c}{\textbf{Meaning}} \\
            \hline
            -1 & Non-existent \\
            0 & In-active (zero energy) \\
            +1 & Active (non-zero energy)
        \end{tabular}
    \end{center}
    \caption{Particle state legend}\label{tab:particle-state}
\end{table}

The code procedure is as follows:
\begin{enumerate}
    \item Photon effects
        \begin{enumerate}
            \item Determine photons in the array that can undergo pair production
            \item Generate pairs of charges, each with half the energy of the pair-producing photon
            \item Add new entries to the charge array
            \item Set the state of the pair-produced photons to 0.
        \end{enumerate}
    \item Charged particle effects
        \begin{enumerate}
            \item Check if the charged particles are in the material (Pb)
            \item Determine which charged particles can undergo bremsstrahlung
            \item Calculate the energy transactions involved in the bremsstrahlung, generate photons
            \item Subtract energy from charged particles due to ionization
            \item Ensure that ionization loss does not exceed the energy of a charged particle
        \end{enumerate}
    \item Repeat until all active photons and charges leave the mass of material
\end{enumerate}

% subsection Monte-Carlo Procedure (end)

% section Implementation (end)
\section{Results} % (fold)
\label{sec:Results}

\iffalse
\begin{figure}[htpb]
    \begin{center}
        \includegraphics[width=0.95\textwidth]{figures/charged_depth_hist.pdf}
    \end{center}
    \caption{Histogram of charged particle radiation lengths}\label{fig:hist}
\end{figure}
\fi

This analysis looks at results from 100 trials. In Figure (\ref{fig:results}), histogram counts were averaged over 100 runs. Curve-fitting was applied to the histogram bin counts, assuming each bin count had radiation length depths (in \unit{\radlen}) corresponding to the center of each bin. 

\iffalse
The curve used to fit the histogram counts had the functional form
\begin{equation}
    N(x) = A x^b \exp \left(-\frac{x}{c}\right)
\end{equation}
\fi

\begin{figure}[htpb]
    \begin{center}
        \includegraphics[width=0.95\textwidth]{figures/charged_depth_results.pdf}
    \end{center}
    \caption{Histogram of charged particle radiation lengths with Poisson errors and a fit curve. Poisson errors were scaled down for visibility.}\label{fig:results}
\end{figure}

\iffalse
\begin{figure}
    \begin{center}
        \includegraphics[width=0.95\textwidth]{figures/fit_parameter_convergence.pdf}
    \end{center}
    \caption{Curve-fitting parameters plotted over varying number of bins}\label{fig:convergence}
\end{figure}
\fi

\begin{figure}
    \begin{center}
        \includegraphics[width=0.95\textwidth]{figures/shower_area_vs_energy.pdf}
    \end{center}
    \caption{Shower curve area relation with initial photon energy \( E_\gamma \).}\label{fig:shower-area}
\end{figure}

The energy of the initial photon is estimated by the following relation
\begin{equation}
    E_\gamma \propto \sum_{i} N(x_i) \Delta x_i
\end{equation}
Figure (\ref{fig:shower-area}) supports the idea that shower curve area can reliably estimate the initial energy of the high-energy photon \( E_\gamma \).

The expected peak in the charged particle shower curve is estimated by the relation
\begin{equation}
    x_\text{peak} = \frac{\ln \left( E_{0} / E_c \right) }{\ln \left( 2 \right) } \cdot X_{0}
\end{equation}
Figure (\ref{fig:results}) has a peak in the shower curve that seems to agree with the expected peak. It is also worth noting the second hill in the shower curve after \( 10X_{0} \). The second hill forms as a result of the remaining charged particles after passing through the mass of lead.

Figure (\ref{fig:shower-area}) demonstrates the behavior of calculating initial photon energies from ionization losses. One can estimate the initial photon energy by calculating the total radiation length traveled by all charged particles in the material. This can be determined by integrating the shower curve in Figure (\ref{fig:results}) to get the total radiation length by all charged particles. Multiplying the total radiation length by the ionization loss rate (\SI{2.0}{\MeV\per\radlen}) will yield a reliable estimate for the original photon energy for low energies. Lower incoming photon energies will produce less energetic charged particles, which are less likely to escape the material before succumbing to ionization losses. Thus, the total energy is more likely to end up in ionization losses. Figure (\ref{fig:shower-area}) supports this explanation, as the calculated ionization loss for incoming photons at \SI{1000}{\MeV} or less fall almost precisely on the orange ``energy conservation line.''

Accordingly, this method begins to diverge at higher photon energies. As more cascading charged particles have enough energy to escape the material, an increasing fraction of the total energy is carried away as charged particle kinetic energy or escaped photons. This is exemplified by how the ionization losses curve below the orange line at higher incoming photon energies. The gap between the blue and orange line signifies the losses due to escaped charged particles and photons.

Increasing the thickness of the lead mass will improve the accuracy of estimating the initial photon energy from the shower curve area. More charged particles will succumb to bremsstrahlung and ionization losses at higher incoming photon energies due to the increased thickness. This will increase size of the region where ionization losses fall near or on the energy conservation line.
% section Results (end)
\section{Conclusions} % (fold)
\label{sec:Conclusions}

% section Conclusions (end) 
\section{Appendix} % (fold)
\label{sec:Appendix}

\newpage
\subsection{Exponential Distribution Code} % (fold)
\label{sub:Exponential Distribution Code}
\begin{verbatim}
import numpy as np
import matplotlib.pyplot as plt

# sample 1000 points from a uniform distribution
uni_sample = np.random.rand(1000)
transform = lambda u, a: -np.log(u)/a

plt.hist(transform_u(sampled, 1), density=true)
plt.plot(x, np.exp(-x))
\end{verbatim}

% subsection Exponential Distribution Code (end)
\subsection{Monte-Carlo Code} % (fold)
\label{sub:Monte-Carlo Code}

% subsection Monte-Carlo Code (end)
\bibliographystyle{abbrvnat}
\bibliography{citations}
% section Appendix (end)
\end{document}
