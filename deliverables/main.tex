\documentclass{article}

\usepackage{graphicx}
\usepackage[margin=1.0in]{geometry}
\usepackage[version=4]{mhchem}
\usepackage{siunitx}
\usepackage{amsmath}
\usepackage{listings}
\usepackage[square,numbers]{natbib}
\usepackage{hyperref}
\usepackage{booktabs}
\hypersetup{
    colorlinks=true,
    linkcolor=blue,
    filecolor=magenta,      
    urlcolor=cyan,
    pdftitle={Overleaf Example},
    pdfpagemode=FullScreen,
    }

\newcommand{\radlen}{\gram\per\cm^2}

\title{\vspace{-3em}{\bf Studying High-Energy Particle Cascades with One-Dimensional Monte-Carlo Simulations}}
\author{Matt Ketkaroonkul}
\date{\today}

\begin{document}
\maketitle

\begin{abstract}
    Write the abstract last.
    
\end{abstract}

\section{Introduction} % (fold)
\label{sec:Introduction}
Particle cascades are a topic of great interest in high energy physics. They occur as cosmic rays, collider interactions, and more.

This project will model a one-dimensional particle cascade of a \SI{1.0}{\giga\eV} photon as it interacts with a mass of lead (Pb). As the photon travels through the lead mass, the photon will undergo pair production, where the photon becomes a positron-electron pair. Each particle having half of the precursor photon's total energy. The charged particles are able to interact with the lead atoms themselves, undergoing ionization losses and bremsstrahlung (braking radiation.)

To describe particle cascades within matter, it is useful to define a quantity called radiation length, which has units equivalent to mass density multiplied by distance. In this paper, we will use the units \unit{\radlen}. Radiation length describes the mean length a high-energy particle travels before it loses energy by a factor of \( 1 / e \) for a given material that makes up a uniform medium. Thus, the density is the density of the medium.

Ionization losses scale linearly with radiation length of a given charged particle. On average, we will assume that a charged particle will lose \SI{2.0}{\mega\eV\per\radlen} of lead traveled through. Bremsstrahlung occurs for charged particles above a critical value of energy of \( E_c = \SI{6.9}{\mega\eV}\). A charged particle with an energy above \( E_c \) will lose 

% section Introduction (end)
\section{Implementation} % (fold)
\label{sec:Implementation}
\subsection{Exponential Distribution} % (fold)
\label{sub:Exponential Distribution}
An exponential distribution has the probability distribution function of
\begin{equation}
    P(x|X_{0}) = \frac{1}{X_{0}} \exp \left( -\frac{x}{X_{0}} \right) 
\end{equation}

where \( X_{0} \) is a scale factor, i.e. the radiation length of lead. In code, we will assume that we only have access to a random variable \( x \) sampled from a uniform distribution. We can sample an exponential distribution from a uniform distribution using \textit{Inverse Transform Sampling}, where
\begin{equation}
    x = F^{-1} (y) .
\end{equation}
\( F(x)= e^{-x / X_{0}} / X_{0} \) and \( y \) is the uniform distribution random variable \cite{bohm}. Of course, this principle applies for when \( x, y \) share a one-to-one mapping. We can then plot a histogram depicting this transformed distribution with the code in Section \ref{sub:Exponential Distribution Code},

\begin{figure}[htpb]
    \begin{center}
        \includegraphics[width=0.7\textwidth]{figures/exp_dist.pdf}
    \end{center}
    \caption{Exponential distribution histogram and PDF with \( X_{0} = 1 \).}\label{fig:exp-dist}
\end{figure}

% subsection Exponential Distribution (end)
\subsection{} % (fold)
\label{sub:}

% subsection  (end)

% section Implementation (end)
\section{Results} % (fold)
\label{sec:Results}

\iffalse
\begin{figure}[htpb]
    \begin{center}
        \includegraphics[width=0.95\textwidth]{figures/charged_depth_hist.pdf}
    \end{center}
    \caption{Histogram of charged particle radiation lengths}\label{fig:hist}
\end{figure}
\fi

This analysis looks at results from 100 trials. In Figure (\ref{fig:results}), histogram counts were averaged over 100 runs. Curve-fitting was applied to the histogram bin counts, assuming each bin count had radiation length depths (in \unit{\radlen}) corresponding to the center of each bin. The curve used to fit the histogram counts had the functional form
\begin{equation}
    N(x) = A x^b \exp \left(-\frac{x}{c}\right)
\end{equation}

\begin{figure}[htpb]
    \begin{center}
        \includegraphics[width=0.95\textwidth]{figures/charged_depth_results.pdf}
    \end{center}
    \caption{Histogram of charged particle radiation lengths with Poisson errors and a fit curve. Poisson errors were scaled down for visibility.}\label{fig:results}
\end{figure}

\begin{figure}
    \begin{center}
        \includegraphics[width=0.95\textwidth]{figures/fit_parameter_convergence.pdf}
    \end{center}
    \caption{Curve-fitting parameters plotted over varying number of bins}\label{fig:convergence}
\end{figure}

\begin{figure}
    \begin{center}
        \includegraphics[width=0.95\textwidth]{figures/shower_area_vs_energy.pdf}
    \end{center}
    \caption{Shower curve area relation with initial photon energy \( E_\gamma \).}\label{fig:shower-area}
\end{figure}

The energy of the initial photon is estimated by the following relation
\begin{equation}
    E_\gamma \propto \sum_{i} N(x_i) \Delta x_i
\end{equation}
Figure (\ref{fig:shower-area}) supports the idea that shower curve area can reliably estimate the initial energy of the high-energy photon \( E_\gamma \).
% section Results (end)
\section{Conclusions} % (fold)
\label{sec:Conclusions}

% section Conclusions (end) 
\section{Appendix} % (fold)
\label{sec:Appendix}

\newpage
\subsection{Exponential Distribution Code} % (fold)
\label{sub:Exponential Distribution Code}
\begin{verbatim}
import numpy as np
import matplotlib.pyplot as plt

# sample 1000 points from a uniform distribution
uni_sample = np.random.rand(1000)
transform = lambda u, a: -np.log(u)/a

plt.hist(transform_u(sampled, 1), density=true)
plt.plot(x, np.exp(-x))
\end{verbatim}

% subsection Exponential Distribution Code (end)
\subsection{Monte-Carlo Code} % (fold)
\label{sub:Monte-Carlo Code}

% subsection Monte-Carlo Code (end)
\bibliographystyle{abbrvnat}
\bibliography{citations}
% section Appendix (end)
\end{document}
